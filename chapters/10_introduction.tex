\chapter{Introduction}

Our understanding of built environment and cities in terms of their form and function have evolved significantly since the early 20\textsuperscript{th} century.
What started as an field of research focused on the physical form of the spaces and places later moved towards modelling them as a function of the population that lives in them. 
Rather than viewing the built environment as infrastructure that need to be built, maintained and managed independently, they have been increasingly viewed as the manifestation of the distribution and dynamics of population embedded in them.
This paradigm change further broadened in the later part 20\textsuperscript{th} century to include economic and social activity of the population that happens along the fabric of these built environment.
Moreover, with the dawn of the information age around the turn of the millennium, researchers even began to view the built environment as the tangible result of information exchange that happens beneath them where cities can be seen as high density clusters of information exchange rather than a place with concentration of physical infrastructure such as buildings and roads.
This information revolution has not only changed researchers' understanding of the underlying forces of the built environment, but also changed how they approach the task of measuring, analysing, modelling and managing them.
It provided them with numerous new technologies, methodologies, tools and, most importantly, unprecedented availability of the comprehensive, granular data generated from the fundamental functions of the built environments.
Availability of these data and tools has turned numerous disciplines upside down where the research, rather than taking a top-down `systems' approach, now tackles the problems using bottom-up, `data first' approach. 

We are currently in the age of `data deluge' where the amount of data generated far exceeds the capacity to analyse and make sense of them.
This data deluge has been accelerated to an extent that 90\% of the data generated in the world has been generated in the last 2 years!
This trend is not going to change any time soon.
With people's almost every day to day activity, banking, payment, ticketing, social communications, fitness tracking etc. have been getting digitised the amount of unstructured, free, open, readily available data is unprecedented.
Collecting data for research has change d from a highly structured designed endeavour to  a low cost scraping activity where data about the population are generated and dumped across the world for never to be used making them all `passive' i.e collected without any effort from the side of participants.
There have been a huge trend for big data everyone with such troves of data want to use the 'big data' techniques to extract values from them.
The above two phenomena - Attempt to model the physical environment as a function of information generation/consumption/exchange and the unprecedented availability of data has lead to quite significant volume of research where a variety of data sources have been used to understand variety of aspects of the built environment.
For eg. cycling in cities are studied from the data generated by Docks, functional regions of a country can be identified from phone calls pairs, population and demography studied through social media such as twitter. 

This frenzy of data generation is not without its pitfalls as well. 
In 21st century all these technologies have also become much more personal.
With personal mobile devices being the norm, almost every data point generated has a person behind it.
The rush into the information age and the social media platforms happened at a much faster rate before the privacy ramifications could be understood.
Most of the data generated are personal and potentially sensitive when linked with other data.
For example, cycle rides itself might not be interesting information, but combined with other movement datasets such as taxi trips can disclose the residences of the people who were being surveyed.
This has prompted backlash from users who don't want their data be monetised.
Which propagated in two forms - technology and legislation.
Technology uses cryptography to anonymise, obscure and encrypt personal information. Legislation such as GDPR aims to influence the behaviour of the people who are handling the data. 
Both of them ultimately try to protect user's privacy and personal information from unauthorised access.S
This poses one of the  greatest risks to the above opportunity we identified earlier, almost every freely available data sources have been protected from unauthorised scraping and when that is not possible the data is changed to remove any risk for user privacy.
All the research which were done on data sources which records any personal information will be rendered impossible with both these ways.
It is imperative to find other ways.

In addition to the privacy concerns the deluge of data also introduces technological challenges which are in the forefront of research under the  
The hype of `big data' and the corresponding tools has also a phenomenon in the last decade.
It promises a lot but introduces a lot of overheads as well.
Blindly jumping on the big data bandwagon has the potentially cause more problems than advantages.
GI science has dealt with large datasets before but the complexity and unstructured-ness of the data encountered from these passive data sources. 
There needs to be careful consideration of the tools and frame works which we can use to address the unique requirements of the data sources.

Working within this context, This PhD thesis tries to use the opportunities presented by the data generated by people passively in the information age, solves the problems arising due to the lack of structure and need to protect the privacy of the users  and produce usable information regarding the distribution and dynamics of footfall at a national level.
The potential use of such information is immense. 
Such information forms one of the components in the building of a smart city where the information on the  built environment is recorded and available.
leading to a connected city where real time census of people and their movement is possible. 
We can not only take snapshot of the state of the city, we can record and understand the built environment as living, breathing organism. 
We can even link this information to other sources of data such as consumer datasets, public transport 
The insights we get by combining this information with other similar info is more than sum of their parts. 
It can revolutionise understanding , planning, policy etc, urban management and finally industry such as retail, transportation etc. city mapper, sharing economy etc.
Though the research doesn't go in detail with the applications it aims to serve as a basic step for studies that will explore the use of the data further.
Contribute to the volume of research which use available data to make conclusions.
The toolkit can be transferred to other fields, the methods can be applied to different sets of data the output data can itself used in various fields by various stakeholders in the retail industry,
\begin{enumerate}[leftmargin=2em, rightmargin=2em]
  \item \textit{Retailers} can get fine grained information on when their customers user the shop leading better business optimisations.
  \item \textit{Customers} can get information on popularity of the places and average crowd information.
  \item \textit{Landlords} can get a way to objectively evaluate their property values based on location and also time.
  \item \textit{Local Authorities} can be enabled to monitor and maintain the health of their retail areas over long periods of time. 
\end{enumerate}
This information has immense value in multiple disciplines of research in addition to human geography such as urban planning, emergency management, etc.

To give an overview of the things we discuss in this research.
In Chapter \ref{chapter:literature} a broad and systematic literature survey on the topic of 'distribution and dynamics of human activity' was conducted.
The focus was on two things - The major themes of research and the technologies that were used in the research.
The overall development of the research area over the past 30 years was analysed.
From all this the opportunities available for research were identified setting up the stage for further research.
In chapter \ref{chapter:collection} we look into Wi-Fi in detail.
Look at probe requests and identify relevant information present in them.
Then device few small controlled experiments to study and understand the data.
 we devise a broader data collection for a larger set of test data.
we also introduce smart street sensors - a large project running for 3.5 years.
Finally look at various uncertainties present in the data and possible solutions.In chapter \ref{chapter:processing} three major things - data toolkit to deal with the data, methods to solve the uncertainties and finally combining the two a data architecture or pipeline which works with smart street sensor to produce continuous stream of data.
Chapter \ref{chapter:application} is an assortment of applications and ideas for use of the data we produced. We work within 4 major areas, footfall index, pedestrian flow, event detection and functional hierarchies of places. An finally we conclude with a discussion of the contributions of this work and various ways the research could be taken forward.

