\chapter{Review of Literature}
Understanding the scale, nature and dynamics of distribution of population in space and time has been a central premise of academic research in various fields of study such as human geography, sociology, urban planning and architecture.
This granular knowledge of where people are and how they move is critical in practical decision making in various industries such as real estate for valuing places, retail for business planning and emergencies for risk management and evacuation.
The first challenge in any of these research is collecting precise and accurate data.
This started as research into methods estimating and interpolating highly granular data from existing regional level aggregate datasets and as the technology improved through the later half of twentieth century, the research methodologies adopted the new technologies to utilise these more granular sources of data.
Though new technologies provide immense opportunity in collecting large amounts of data which were previously impossible they also introduce their share of uncertainties.
Hence it becomes imperative that we understand the evolution of these techniques and methodologies along with the research that used them to build our rationale behind any further research.

Moreover with the proliferation of mobile devices and wireless internet connectivity, even day to day activities are being digitised leading to the creation  of large amount of easily accessible data which are generated passively in an unstructured manner.
The users' acceptance to the collection and analysis of such data has also been improving until recently \cite{kobsa2014}.
There has also been rising concerns regarding user privacy along with the development of more accurate methods to track them.
In this context, the critical task in all these research is to solve the problem of balancing these two by choosing the right technologies and devising the appropriate methods.

In this chapter we conduct a systematic survey of literature in this broad area of `distribution and dynamics of human activity'.
The aim of this survey is to evaluate where we are at the research and understand how we got here.
First we conduct a comprehensive survey of over 300 publications which discuss this area of research and we then look at the major themes and trends in the last 70 years.
We discuss these themes in detail looking at the aims and achievement of these research while highlighting the opportunities and gaps they leave.
We also look at the timeline of their publication and discuss the evolution of the research along with the changes in the technology landscape.
We then classify the studies by looking at the technologies employed by them and note the trends in the adoption and phasing out of relevant technologies.
We also try and understand the advantages and disadvantages of these techniques and develop a theoretical framework to understand when and how to use them effectively answer research questions.
Finally we summarise the literature survey focussing on the major gaps still left in our understanding and the new, emerging areas where more research is necessary.
We also critically evaluate these areas of research in terms of priority and feasibility to develop our research questions and devise a plan for our research. 

\begin{marginfigure}
  \includegraphics{images/literature-timeline.png}
  \caption{Growth of research in the areas of understanding distribution and dynamics of human activity since 1980.}
  \label{figure:literature:timeline}
  \vspace{1em}
  \noindent\fontsize{7}{7}\textit{Measured in the number of papers published}
\end{marginfigure}

We begin our literature review with a systematic search for academic publications studying the distribution and dynamics of human activity.
We start this search from a set of reviews pertaining to the topic such as \citet{ratti2006, jiang2013, steenbruggen2013, arribas-bel2014} and \citet{li2016}.
From these reviews we further expand by navigating the citation networks and finding research which are relevant to our search.
Though this does not give us a perfectly comprehensive set of research literature, it does provide us with a representative sample of all the different directions of the research conducted in the area.
Through this process, we identified around 325 relevant research publications which deal with the collection, measurement, analysis, visualisation and discussion of population at a granular level.
The research started around 1950s looking at estimating day-time urban population at a granular level using existing  broader data employing various estimation methods \cite{foley1954, schmitt1956}.
Though this served as a starting point, the pursuit of such granular data and their applications in corresponding fields didn't pick up until the start of the 21st century fuelled by the digital revolution that followed growth of internet.
Figure \ref{figure:literature:timeline} shows the yearly volume of research published since 1980.
We can observe though there were some research conducted through 80s and 90s the real push forward came around beginning of the millennium when mobile phones adoption skyrocketed.
In addition to the early 2000s we also see a substantial increase in interest in the beginning of the next decade fuelled by the smartphone revolution which completely changed the research avenues in-terms of volume and types of data available and methodologies available to tackle them.
The area of research is multidisciplinary encompassing academic interest and commercial applications in Geography, Urban Studies, Urban Planning and Management, Emergency planning and Management, Economics, Computer Science and Engineering etc.


%==============================================================================%
% Section Introduction
%==============================================================================%

\section{Research Themes}
In this section we look at the major themes and questions tackled by this knowledge base.
We start by classifying the research into the major and minor themes explored in them as shown in Figure \ref{figure:literature:themes}.
The tree-map shows the volume of research in corresponding themes measured in terms of number of publications.
We can observe that the research is conducted in five major areas - population studies focussing on the creating and utilising data on distribution and nature of human activity, mobility and interaction focussing on the changes in these distributions, understanding the nature and function of space from these distribution and change, methods and techniques which can be used to conduct the research and finally issues and solutions related to the privacy of the users while conducting these research.
We can also observe that most of the research apart from developing methods were conducted in the domain of human mobility and social interaction closely followed by the population distribution.
In the following sections we discuss these in detail along with their sub themes with the following framework,

\begin{enumerate}
  \setlength{\itemindent}{2em}
  \itemsep-0.25em
  \item What are the major lines of questioning?
  \item What has been done previously?
  \item Where are the opportunities for further research?
\end{enumerate}

\begin{figure}
  \includegraphics{images/literature-themes-treemap.png}
  \caption{Tree-map showing the volume of research conducted under each major themes and their sub-themes.}
  \label{figure:literature:themes}
\end{figure}

%------------------------------------------------------------------------------%
% Population Studies
%------------------------------------------------------------------------------%

\subsection{Population Studies}
Though \citet{foley1954} and \citet{schmitt1956} started this line of research in 1950's with the discussion on estimating daytime population using broader datasets it was not until the 80s significant volume of research kicked off in this area of study.
From 80s until mid 2000's numerous studies were conducted on measuring and studying the population at a granular level both spatially and temporally.
The focus of the research around this time was primarily on interpolation from the larger datasets created using censuses, regional or national level sample surveys and other centrally collected sources of data.
There have been numerous fairly successful attempts with methodologies where a broad dataset such  as regional level population summaries and modelling or interpolating more granular data from them by augmenting with other sources of data such as street networks \citep{reibel2005}, remote sensing \citep{sutton1997, yuan1997, chen2002} etc.
\citet{dobson2000, dobson2003a, bhaduri2002, bhaduri2007} and \citep{mennis2003, mennis2006} are examples of such research methodology.
These studies were almost done on a city scale or above with mostly modelling or interpolation methods since the data sources were few and were centrally collected.

Around 2005, there was a sharp shift in research where the interpolation methods were replaced by highly available granular data collected over cellular network.
Studies were conducted on estimating population densities, presence of tourists, general activity pattens using data from cellular networks.
Most of these research were conducted at a far larger geographic scale looking at things at an area level \citep{pulselli2008,girardin2009,phithakkitnukoon2010,yuan2016}.
There were efforts in using device level sensors such as global positioning system(GPS), Wi-Fi and Bluetooth to detect population distribution and socio-geographic routines \citep{calabrese2010,rose2010,farrahi2010}.
There have been studies on looking at people distribution as granular as queue lengths as discussed by \citep{wang2013} to city level dynamic population mapping where the limitations of traditional datasets generated through censuses and surveys \cite{deville2014}.

Around the 2015, along with the data collected directly from the mobile devices,the data that are generated by the users activity on these devices are became more important.
Social media data such as twitter \citep{lansley2016} and other consumer data such as loyalty cards \citep{lloyd2018}, smart cards \citep{ordonez2012} etc. have also become a significant sources of data for such research.
Recently, with increased concerns and legislation on privacy, there have been studies which go back to the effort of interpolating granular data from broader datasets but using more data and processor intensive technologies such as agent based modelling, deep learning, small area estimation \citep{crols2019, shibata2019, rao2015} etc..
Though there have been a lot of work done in most of the directions in this research area, the clear gap arises due to the absence of a continuous, granular and sufficiently longitudinal data-sets to complement the methodologies that have been developed. 

%------------------------------------------------------------------------------%
% Mobility and Interaction
%------------------------------------------------------------------------------%

\subsection{Human Mobility and Interaction}

This is one of the major areas of research which have significantly benefited from the decentralised collection of data at a granular level \cite{castells2000}.
In addition to being useful in their own right, these data were in turn used to augment traditional models of travel behaviour, traffic and transport to provide a better understanding of human movement over time and space \citep{janssens2013}.
The major themes of research within this area are, Movement of people in space and time with emphasis on understanding the built environment, social interaction between these people with a sociology perspective and traffic and transportation studies with a infrastructure perspective.
There is significant volume of research which dealt with recording and analysing the trajectories of the users to understand their movement patterns enabled by the unprecedented availability of detailed data from mobile devices.

%
% \begin{enumerate}
%   \setlength{\itemindent}{2em}
%   \itemsep-0.05em
%   \item 
%   \item 
%   \item 
% \end{enumerate}
%



%------------------------------------------------------------------------------%
% Methodology and Techniques
%------------------------------------------------------------------------------%

\subsection{Methodology and Techniques}

\citep{maceachren2001}
\citep{hallisey2005}
\citep{morrison2000}
\citep{lobben2003}
\citep{harrower2007}
\citep{ferrara2014}
\citep{fabrikant2005}
\citep{thomas2005}

Visualising the temporal dynamics of data collected on human activities through decentralised processes poses significant challenges when approached with traditional cartographic concepts (MacEachren, 2001 Hallisey, 2005).
Digital media especially animation has been explored as an option to solve for the temporal dimension (Morrison, 2000; Lobben, 2003) but is bound by the cognitive limits of the viewer (Harrower, 2007).
There have been approaches proposed around animations of generated surfaces (Kobayashi, 2011) and network-based visualizations (Ferrara, 2014) leaving gaps in research for new methods in dynamic geographic visualisation (Fabrikant, 2005) and visualising path and flow of phenomena (Thomas, 2005).
This provides us with a promising opportunity for research in methods for visualising high frequency, hyper-local pedestrian data within the limits of cognition of the viewer.

%------------------------------------------------------------------------------%
% Spatial Analysis Theory and Modelling
%------------------------------------------------------------------------------%
\subsection{Spatial Analysis - Theory and Modelling}

Traditional and modern geography was dominated by the study of centrally collected data acquired through extensive field surveys and remote sensing.
In the last two decades, a significant paradigm change has been introduced by the availability of unprecedented amount of data generated by unconventional sources such as mobile phones, social media posts etc.
This move to the postmodern geography has been accompanied by a change in our understanding of the built environment and human geography from a static point of view to a more dynamic definition \cite{soja1989}.
This definition is based on the bottom-up mechanisms which make human activity such as information exchange and economy to manifest in the physical built environments as argued by \citep{batty1990, batty1997, batty2012} and \citep{batty2013, batty2013a}.

This transition into the digital age \citep{graham1999, tranos2012, tranos2013} has changed the politics of space and time \citep{massey1992} and been more pronounced in the study of urban built environment where technology has redefined the concepts of place and space \citep{graham2001, graham2002, sassen2001}.
With the ability to collect and analyse of data on large complex systems in real-time \citep{graham1997}, we are exploring the possibilities of understanding their structure and organisation using concepts of complexity theory \citep{bettencourt2013, portugali2012} with more emphasis on their temporal patterns such as the argument towards finding the pulse of the city \citep{batty2010}.
With the population getting more and more connected \citep{castells2010}, the nature of space/place is being dynamically defined by the population themselves \citep{giuliano1991} and vice versa \citep{zandvliet2006}.
This flood of hard data \cite{nature2008} was accompanied not only by optimism in its potential \citep{thomas2001} but also by the questions raised on the challenges in handling the diverse, large scale, non standardised data it produces and the usefulness or representativeness of the resulting analysis \citep{miller2010, arribas-bel2014a}.

However, availability of such data has impressive uses in urban studies \citep{bettencourt2014} especially with advancement of new technologies \citep{steenbruggen2015} and possibility of distributed, crowdsourced data collection \citep{lokanathan2015}.

%------------------------------------------------------------------------------%
% Privacy
%------------------------------------------------------------------------------%
\subsection{Privacy}

The ubiquity of personal devices and digitisation of day to day activities through these mobile devices \citep{mcmeel2018dark} has provided many opportunities for researchers and industry for collecting, analysing and deriving inputs from them.
However at the same this also increased the risk of infringement on privacy of the users whose data is being collected \cite{saponas2007, krumm2009}.
There is immense value in uniquely identifying and profiling information on people for specialised purposes such as security \citep{cutter2006} and law enforcement \citep{dobson2003} but also has extreme risks associated when not handled with care \citep{vanwey2005}.

Strictly protecting personal information while ensuring the information is usable for research by maintaining the uniqueness in the data is the major concern which was addressed by devising frameworks for secure practices in confidentially collecting and using the location data \citep{duckham2006, tang2006, lane2014}.
Some efforts sought to accomplish this task through cryptographic hashing algorithms (Pang, 2007) while others aimed to thwart identification and tracking at the device level by techniques such as MAC randomisation \citep{gruteser2005, green2008}.
Finally though getting consent of users for the collection and use of such information from their mobile devices is challenging, there is a significantly improved acceptance when the process offers value in return such as discounts and monetary benefits \citep{kobsa2014user}.

There is opportunity in this area for research in applying the cryptographic solutions along with the privacy preserving frameworks to arrive at methods which can extract useful information out of large personal data while obscuring or anonymising them.

%==============================================================================%
% Research Trends
%==============================================================================%

\begin{figure*}
  \includegraphics{images/literature-themes-timeline.png}
  \caption{Outline of the `Medium data toolkit' devised to collect, process, visualise and manage the Wi-Fi probe requests data}
  \label{figure:literature:themes:timeline}
\end{figure*}

\section{Research Trends}

Figure \ref{figure:literature:themes:timeline} shows the volume of research done in this topic since 1980 categorised based on their major themes discussed earlier.
We can observe that there are distinct trends in the research over time, which evolved around the development of technology in the last two decades.
Until 90s the research was mostly centered around population studies on estimating and interpolating granular spatial and temporal information from larger and cross sectional datasets such as census and sample surveys.
The period between 2000-2010 there was interest in potential of the new data generated by the digital revolution. 
We can categorise this as the `mobile era' where carrying mobile devices become mainstream.
This explosion of research coincided with mobile phones becoming more popular and ubiquitous with population in urban areas and was around development of methods and techniques to utilise the data generated from them.
There were also extensive studies in using the datasets to understand human mobility along with a rising concern in the privacy of the users who's data which are being used for these studies.

%------------------------------------------------------------------------------%

The release of iPhone in 2008 and the increase in the share of 'smartphones' in the next 10 years sparked the `smartphone' era. 
The change made sure that all the mobile devices gaining numerous capabilities such as internet connectivity over Wi-Fi and mobile network, location awareness with global positioning system, movement recognition with accelerometers and connectivity other `wearable' devices through Bluetooth.
This also lead to the digitisation of lifestyle where every aspect of the life being done through these devices over internet while generating huge amount of data on these activities.
This sparked the large volume of research on the form and function of space by studying this data and on the dynamics of human population in space and time in the next 5 years.
These research were particularly centered around tracking the trajectory of people using the mobile devices they carry with them as the smartphones made it easier to collect the necessary data directly from them rather than depending on a centrally collected datasets from mobile carriers. 
With the theoretical limit to predictability in human mobility quantified \cite{song2010limits}, the focus on urban mobility has been declining in the past few years which has led to a renewed interest in population studies at a local-local level in real-time.
In addition to using the data from the mobile devices, these studies have also been exploring the use of large assemblages of consumer data that are being generated in this connected mobile environment and linking them together to create a fuller picture \cite{cdrc2018}

%------------------------------------------------------------------------------%

Finally, with the increase in use of personal data, there has also been an increase in research regarding the privacy of the users.
Along with this, the mobile devices and subsequently the data generated by them are more and more anonymised so that the users cannot be tracked or identified at a personal level.
This has given rise to the new trend in research to devise methods to overcome this anonymisation and at the same time research which considers these methods as vulnerabilities and find solutions to make the anonymisation process more robust. 
There is clear need for methods which anonymise the data sufficiently to protect the identity of the users and at the same time enable us to conduct research in
measuring studying population distribution and movement at a granular level.

%==============================================================================%


\section{Techniques and technology}

When we look at the literature from the technology perspective, we observe that the continuous application of recent technological developments in the pursuit of understanding the distribution of human activity and population spatially and temporally over the past two decades.
The distribution of the research in terms of the main technique/ technology used over the years is shown in Figure 1.
 and the total distribution is shown in Figure 1.
. We observe that the earliest attempts started from the exploration of using interpolation and modelling techniques on the available coarse data and as the need for more granular datasets increased there were attempts to devise and utlize bespoke solutions to generate them.
When mobile devices became mainstream, the focus shifted to utilize the relevant components of the mobile infrastructure.
A significant number of studies were done in utilising data collected from the mobile network, sensors in the mobile devices, especially GPS and WiFi, in addition to the social media content generated from these devices.
A detailed account of these studies is given below,


\begin{marginfigure}
  \includegraphics[trim={1.1cm 1cm 1cm 1cm},clip]{images/literature-technology.png}
  \caption{Growth of research in the topic of `'}
  \label{figure:literature:timeline}
\end{marginfigure}
\marginnote[1em]{\fontsize{7}{7}\textit{}}


\subsection{Interpolation and Modelling}
Attempts in using the existing data collected through traditional methods such as census and large scale sample surveys to create spatially and temporally granular and detailed estimates were carried out by applying various interpolation methods such as pycnophylactic, dasymetric interpolation (Tobler, 1979; Mennis, 2003; Mennis, 2006; Hawley, 2005; Tapp, 2010) along with spatial (Lam, 1983; Martin, 1989) and temporal interpolation techniques (Glickman, 1986).
These methods along with supplementary data such as remote sensing imagery (Sutton, 2001; Chen, 2002) and street networks (Reibel, 2005) were shown to be useful in producing detailed granular population maps at various scales with varying degree of success (Dobson, 2000; Bhaduri, 2002; Dobson, 2003; Bhaduri, 2005; Bhaduri, 2007).
These approaches have been employed in various applications such as econometric studies (McDonald, 1989), studies on public health (Hay, 2005), emergency management (Kwan, 2005) and flood risk estimations (Smith, 2016).
In addition to these interpolation techniques classic modelling techniques can also be used to estimate daytime populations and demographic structure at hyper local scales (Jochem, 2013; Jia, 2014), urban scales (Alahmadi, 2013; Abowd, 2004) and regional scales (Foley, 1954; Schmitt, 1956, Singleton, 2015).
The granular data created with such modelling techniques are shown to be useful in urban planning and management (Parrott, 1999), emergency management (Alexander, 2002; Cutter, 2006) and in modelling traffic and transportation (Lefebvre, 2013).
These interpolation and modelling techniques along with granular data produced are also used in classifying spatial areas and hence understanding the structure of cities in general (McMillen, 2001; McMillen, 2004; Lee, 2007; Arribas-Bel, 2014 b).
Though being useful, these techniques are still shown to have limitations and uncertainties (Nagle, 2014), which mostly arise from the nature of the input data employed.
This leads us to the need for more detailed and frequent collection of data.

\subsection{Bespoke technologies}

Following this need, there has been efforts to use bespoke/specialised technologies such as cameras (Cai, 1996 Heikkilä, 2004 Kröckel, 2012), Lasers (Zhao, 2005 Arras, 2008) and radio frequency receivers  (Bahl, 2000 Yang, 2013 Chothia, 2010 Bulusu, 2000 Dil, 2011) to measure human activity.
But the major problem with such solutions is the cost and effort involved in implementing them at large scales.
Moreover, being specialised and centralised they tend to be challenging to maintain and update.
In the addition, the rise of mobile phones as ubiquitous personal devices for the broader population has made them a viable alternative for collecting such data with greater granularity at large scales.

Mobile infrastructure consists of both the ‘network part’, built and managed by the service providers, and the ‘user part’, which is the phones owned by the users’ themselves.
The network part, in addition to providing connectivity to the users, also collects information on these devices actively (calls, messages) and passively (tower to tower handover).
The mobile devices themselves have a variety of sensors (accelerometer, compass, barometer etc) and capabilities (cellular, WiFi, NFC etc) that can be sources of data themselves.
With the growth of mobile devices and the infrastructure surrounding it, there has been significant effort in utilising data generated by every component of this complex infrastructure.

\subsection{Cellular/ Mobile network}

The use of cellular network data is relevant for urban studies (see Jiang, 2013; Steenbruggen, 2015; Lokanathan, 2015; Calabrese, 2015; Reades, 2007) even though it is acknowledged to have inherent biases such as ownership bias across particular demographic groups (Wesolowski, 2013).
Visual exploration of use of such data using interactive interfaces to evaluate quality of service and scenario testing has been tested for the optimisation of public transport (Sbodio, 2014).
Such network data with the active and passive information collected from them can be used to create trajectories of people (Schlaich, 2010), detect their daily routine (Sevtsuk, 2010) and classify those routes (Becker, 2011).
It was also demonstrated to be useful in understanding overall mobility and flow of people and information (Candia, 2008; Krings, 2009; Simini, 2012; Zhong, 2016).
It can be used to identify asymmetry in flow of people spatially (Phithakkitnukoon, 2011), estimate volume and pattern of road usage (Bolla, 2000; Wang, 2012) and by augmenting the topology to optimise operations (Puzis, 2013).
Such datasets have been extensively used in traffic and transportation research to derive origin-destination matrices (Caceres, 2007; Mellegard, 2011; Iqbal, 2014), travel time estimation (Janecek, 2012) and traffic status estimation (Demissie, 2013; Grauwin, 2015)        
It has been shown that mobile network data can be used to uncover nature of the population such as tourists in specific areas (Girardin, 2008) and the interaction between the people in the study area.
The structure (Onnela, 2007 a, b) geography (Lambiotte, 2008) and dynamics (Hidalgo, 2008) of such networks have been studied and demonstrated to be useful in predicting their change (Wang, 2011).
The network data and its spatio-temporal structure can also be used for classification of land use (Pei, 2014), assessment of spatial patterns (Reades, 2009; Steenbruggen, 2013) and understanding the spatial structure of cities (Louail, 2014; Arribas-Bel, 2015).
The data collected from the cellular network measured at the smallest scales such as web chatting, mobile calls and so on can be used to create estimations of micro site level population density (Pulselli, 2008), characteristics (Girardin, 2009) and the nature of the activity (Phithakkitnukoon, 2010).
Aggregated human activity measured from the data can be used to measure and model population dynamics and land use density and mix at large scales (Jacobs-Crisioni, 2014; Tranos, 2015).
The spatial patterns understood can then be applied to urban planning (Becker, 2011) whilst the temporal patterns have particular utility for the likes of epidemiology where population influxes measured from changes in mobile network usage can be used to model spread of diseases (Buckee, 2015).

Though the mobile network provides much more granular and accurate data than interpolation techniques, it is not without its limitations.
The network distribution usually follows the purpose of service coverage and commercial decisions which introduces systematic biases in the data passively collected through them, while the data actively collected through them has bias based on the volume of usage of services by the customers which can vary widely based on location and demography.
This makes collection of data directly from the devices using the sensors available a more robust option.

\subsection{Mobile Sensors}

The major sensors and capabilities present in mobile devices that can be used for distributed urban sensing are cellular radio, Bluetooth, WiFi, GPS, accelerometer and a compass.
Since cellular radio is managed by the cellular network and covered in mobile network data, we explore the research done with other sensors.
In contrast to planned actively collected data, data passively collected via a distributed network of general purpose devices tends to be larger and more temporally dynamic.
For example, an organised survey conducted every month to understand interpersonal communications between people in a team of 50 will result in a 2500 records a month.
The same task is done through collecting data on email communication sent by them will result in a same volume records in a day.
The challenges and solutions on collecting and analysing such large-scale longitudinal data are discussed by (Laurila, 2012; Antonic, 2013).
The real time nature of such data also gives us the opportunity to monitor and understand the city in much smaller temporal scales (Townsend, 2000; O'Neill, 2006) and the representativeness of such datasets have also been explored (Shin 2013; Kobus 2013).
Data generated from communication networks can be used to understand the structure of urban systems which are becoming increasingly borderless (Bertolini, 2003).
Similar to the network based data, it can help in understanding human mobility (Asgari, 2013; Amini, 2014; Zhang, 2014) through mining trajectory patterns (Giannotti, 2007) and socio geographic routines (Farrahi, 2010).
It is also useful in various traffic and transportation applications for monitoring roads (Mohan, 2008) and estimating traffic (Cheng, 2006), uncovering regional characteristics (Chi, 2014) and extracting land use patterns (Shimosaka, 2014).
Apart from GPS and WiFi, there have been efforts in exploring other possibilities such as Bluetooth for location (Bandara, 2004) and aggregate detected Bluetooth activity to monitor freeway status (Haghani, 2010).
There have also been successful implementations of frameworks to predict movement of people by combining WiFi and Bluetooth (Vu, 2011).
But owing to shorter range and requirement of active engagement from the user (device pairing) Bluetooth is much less preferable for large-scale data collection than GPS/ WiFi.
The research on GPS and WiFi based studies are discussed in more detail below.

\subsection{Global Positioning System}
In addition to providing a user’s location to applications such as Google Maps, the GPS capability in mobile devices working with the WiFi can maintain a continuous list of locations visited by the device over long periods of time.
It works mostly in the background and requires almost no active input from the user to operate.
Though very convenient for collecting data, due to the privacy risks associated with it GPS is often one of the resources in a device that requires explicit user permission to be accessed.
The concepts and methodologies for collecting such data were set out by Asakura (2004) and there have been attempts to collect this rich data from volunteers at a large scale along with ancillary data (Kiukkonen, 2010) and provide a location based service application for the collection of data (Ratti, 2006; Jiang, 2006; Ahas, 2005).
 
GPS is one of the most used technologies for mobility studies.
It has been used to analyse and understand individual mobility patterns (Gonzalez, 2008; Neuhaus, 2009), which have been shown to have a high order of regularity in spite of the complexity (Brockmann, 2006; Song, 2010 b).
There have been efforts to use this regularity to predict the future location of people (Monreale, 2009; Calabrese, 2010).
The limitations of predictions have also been quantified (Song, 2010a).
There have been successful efforts in extracting behaviours and patterns from such trajectory data (Liu, 2010; Cho, 2011; Hoteit, 2013; Pappalardo, 2013) along to understand individual patterns from large assemblages (Giannotti, 2011; Calabrese, 2013) and vice versa (Wirz, 2012).
In traffic and transportation, GPS trajectory from mobile devices is used to estimate (Calabrese, 2011) and expand (Jing, 2011) OD matrices, detect the mode of travel (Gong, 2012; Rossi, 2015) and calibrate existing spatial interaction models (Yue, 2012)
.
Since the data is collected at the device level and depends on the activity of the individual, it can be de-anonymised to reveal the nature of the owner of the devices.
The possibilities of detecting the activity of the individual from trajectory information is demonstrated by (Liao, 2006; Krumm, 2007).
Patterns (Jiang, 2012) and structures in routines (Eagle, 2009) can be extracted from these trajectories and can be used for socio geographic analysis of the population (Licoppe, 2008).
It can also utilised in classification of the population at a particular location at a given time (Pappalardo, 2015).
Being inherently spatial and activity driven, GPS trajectories have been shown to be useful to identify (Bao, 2012), characterise (Wan, 2013) and automatically label (Do, 2014) significant places of interest.
It can also be used for land use detection (Toole, 2012), classification (Jiang, 2015) and the study of urban morphology (Kang, 2012).
These GPS trajectories have been shown to be useful in estimating population dynamics at local level and within short durations during social events (Calabrese, 2010; Kim, 2014; Deville, 2014).
When combined with other data sources can be useful to understand relationship between spatial areas (Long, 2015).

From the literature we see that GPS is one of the most precise and accurate user side methods of collecting location of mobile devices.
In addition, the data collected is well understood and collection methodologies can be scaled up with minimum resources.
That said, it is well known that urban sensing methods using GPS of mobile devices also has problems of enhanced risk of breach of privacy when done passively and need for user engagement when done actively.

\subsection{Wi-Fi}

WiFi is a wireless network connection protocol standardised by IEEE, 2013.
It is a distributed server-client based system where the client connects to access points (AP).
Every device in the network has a unique hardware specific MAC address, which is transmitted between the device and AP before the connection is made.
The key feature of WiFi infrastructure is that the network is distributed and the APs can be set up and operated by anyone locally unlike mobile networks.
Since they are primarily used for Internet service provision, the protocol has priority for continuity of connectivity so the devices constantly scan for new and better connections.
This is done through a probe request, which is detailed in later sections.
With this background we can see that WiFi provides a fair middle ground between an entirely network driven approach such as cellular network to an entirely user driven approach such as GPS.
Since the network infrastructure is distributed and deployed for Internet it offers near complete coverage, is very resilient,  and can encapsulate and reinforce civic space in cities (Torrens, 2008).

hough WiFi is a location less technology, there are reliable methods to triangulate the location of the device by the signal strength and the known locations of APs (He, 2003; Moore, 2004; LaMarca, 2005).
This can overcome the usual shortcoming of GPS, which struggles for precision and accuracy in indoor and densely built environments (Zalampas, 2006; Kawaguchi, 2009; Xi, 2010).
Utilising this, we can easily and quickly estimate trajectories of the mobile devices just using the WiFi communication the device has with multiple known APs (Sorensen, 2006).
This can be used similar to the GPS trajectories to understand individual travel patterns (Kim, 2006; Rekimoto, 2007; Sapiezynska, 2015), crowd behaviour (Abedi, 2013; Mowafi, 2013), vehicular (Lu, 2010) and pedestrian movement (Xu, 2013; Fukuzaki, 2014; Wang, 2016).
It can also be used in transportation planning and management to estimate travel time (Musa, 2011) and real time traffic monitoring (Abbott-Jard, 2013).

Being a general network protocol designed to be used by mobile devices, WiFi devices relay a range of public signals known as probe request frames on regular intervals throughout its operation, for the purpose of connecting and maintaining a reliable and secure connection for the mobile device (Freudiger, 2015).
These signals can be captured using inexpensive customised hardware, non-intrusively and in turn to be used for numerous applications.
In addition to a uniquely identifiable MAC address, these signals include a range of other information which when combined with the temporal signatures of the signals received can help us understand the nature and identify the devices which are generating these signals.
These device/user fingerprinting techniques are demonstrated by Franklin (2006) and Pang (2007) and the unique MAC addresses and associated information can successfully track people across access points (Cunche, 2014a), their trajectories (Musa, 2012), the relationship between them (Cheng, 2012 Barbera, 2013 Cunche, 2014b) and predict which of them will be most likely to meet again (Cunche, 2012).
Using the semantic information present in these probe requests it is possible to understand the nature of these users at a large scale (Di Luzio, 2016).
Using the received signal strengths from pre placed devices we can monitor the presence and movement of entities that are not even carrying a WiFi enabled device (Elgohary, 2013).

Because of the security and privacy risks posed by the WiFi protocol’s use of hardware based MAC address, various methods to strengthen the security have been proposed (Pang, 2007; Greenstein, 2008).
The randomisation of MAC addresses has become more mainstream in mobile devices with the introduction of it as a default operating system behaviour in iOS 8 by Apple Inc.
Since MAC randomisation is not a perfect solution (Cunche, 2016) there have been numerous attempts to fingerprint unique devices from the randomised anonymous information present in the probe request frames for the purposes of trajectory tracking and access point security.
The methods used are decomposition of OUIs where detailed device model information is estimated by analysing an already known dataset of OUIs (Martin, 2016); Scrambler attack where a small part of the physical layer specification for WiFi is used (Bloessl, 2015); and finally, the timing attack where the packet sequence information present in the probe request frame is used (Matte, 2016; Cheng, 2016).
A combination of these methodologies has been proven to produce de-anonymised unique device information from randomised MAC addresses (Vanhoef, 2016).
In addition to tracking, WiFi probe requests can be aggregated to uncover the urban wireless landscape (Rose, 2010) and used to reveal human activity at large scales (Qin, 2013), pedestrian numbers in crowds (Schauer, 2014; Fukuzaki, 2015) and also counting people in hyper local scales such as queues (Wang, 2013).
With enough infrastructure we can aim to generate a real-time census of the city (Kontokosta, 2016) and also predict the amount of time a device will spend around the sensor as well (Manweiler, 2013).
Similar to GPS data this can be used as an additional control layer for interpolation techniques such as map merging (Erinc, 2013).
A comparison of various approaches was done by Pinelli (2015) where through experiments on a telecom operator dataset, it was showed that using network-driven mobile phone location data is more advantageous compared to the widely used event-driven ones.

\subsection{Social Media}

In addition to the direct data from the sensors themselves the content generated from the mobile devices in social media can provide a viable proxy for estimating the level and nature of human activity.
The use of geo located tweets on the study of small area dynamic population estimation (Ordonez, 2012; Marchetti, 2015; McKenzie, 2015), geotemporal demographics (Bawa-Cavia, 2011; Longley, 2015; Lansley, 2016) and global mobility (Hawelka, 2014) has been thoroughly explored.
These data sources are shown to be useful in social sciences (Crane, 2008), abnormal event detection (Chae, 2012) and analysing urban environments (Sagl, 2012).
It can also be used as a control layer for interpolation techniques we discussed earlier (Lin, 2015).




\section{Research Gaps and Opportunities}

In this section we summarise the previous sections to find out the best possible  technology for further research and discuss the research gaps and opportunities available to us.
Table \ref{table:literature:technologies} summarises the above discussion to evaluate all the relevant technologies that can be used for the data collection and analysis for the study of human activity at a granular level.

\begin{table}
  \footnotesize
  \begin{center}
    \begin{tabular}{p{2.25cm}p{1.5cm}p{1.3cm}p{1.3cm}p{1.3cm}p{1.3cm}}
      \toprule
        Technology & Interpolation & Bespoke & Cellular & GPS & Wi-Fi \\
      \midrule
        Coverage* & Local & City & All & Local & All\\
        \addlinespace[0.2cm]
        Certainty* & Very Low & High & Medium & High & Medium \\
        \addlinespace[0.2cm]
        Independence* & Low & Very High & Low & Medium & High \\
        \addlinespace[0.2cm]
        Intrusiveness* & Low & Medium & High & High & Medium \\
        \addlinespace[0.2cm]
        Granularity* & Very Low & Very High & Medium & High & High \\
        \addlinespace[0.2cm]
        Ease of Collection* & Medium & Low & Medium & Low & High \\
        \addlinespace[0.2cm]
        Scalability* & Medium & Low & High & Medium & High \\
      \bottomrule
    \end{tabular}
  \end{center}
  \caption{Evaluation of different technologies or approaches that can be used for data collection.}
  \label{table:literature:technologies}
\end{table}
\marginnote[-3.75cm]{\textit{* coverage - the density and extent of the current infrastructure. Certainty - the lack of uncertainty in the data. Independence -How much the technique depends on secondary data. Intrusiveness - the potential for infringement of users' privacy. Granularity - the smallest spatial and temporal at which data could be collected. Ease of Collection - how efficient it is collect data in terms of time and resources. Scalability - the potential for the technology to improve coverage.}}

We can observer that Wi-Fi offers the best possible technology in terms of flexibility and scalability for data collection through mobile devices at an individual level while posing some risk to privacy of participants and involves uncertainty regarding the field of measurement.

\citep{pinelli2015} looks at a comparison of various approaches of collecting and analysing mobile phone location data.
The research identifies two major approaches in collecting device location data - Event-driven and Network-driven.
The event-driven approach is centered around the mobile devices generating data 
through their day to day activities.
The major sources of event-driven data are Call Detail Records(CDR) and internet use.
Network-driven approach is centered around the service provision infrastructure such as cellphone towers, Wi-Fi base stations etc.
The methods used to collect network-driven data are periodic update - where the device sends an update stating the base station it is connected to, handover - where the device information is recorded as they are moving between base stations and location update - where the location of the device is recorded based on the base stations it is connected to. 
The research used a set of anonymised mobile phone location data from a Belgian telecom operator for the city of Mons from which various event-driven and network driven scenarios were simulated. 
The authors compared these simulated scenarios for application-independent and application-dependent cases such as spatial dispersal, classes of users, count estimation and flow estimation to understand their relative advantages and disadvantages.
Through these comparisons it was shown that using network-driven mobile phone location data is more advantageous compared to the widely used event-driven ones.

From the literature search we can summarise that there is a considerable opportunity in the collection and analysis of mobile phone based data for measuring hyper-local, spatio-temporal dynamics of human activity.
The potential for research gaps are discussed in detail in the following sections.

%==============================================================================%
\subsection{Ambient population analysis}
%==============================================================================%

\marginnote{\textit{\textbf{Opportunity 1:} Design and collection of national/regional, longitudinal, grass root level data set which enables study of population both spatially and temporally.} }

Previous research in this area of study has been limited to either national/ regional level studies using centrally collected residential population data such as censuses or to area level studies conducted with mobile devices based technologies.
Though there were some efforts in collecting and using mobile phone data at national/ regional level we have never been presented with such unprecedented level of data available now.

For example, \citep{qin2013} demonstrate that it is possible to detect and quantify human presence at locations using probe requests with a detection rate of 86\%. 
Along with the evaluation of the various algorithms for channel switching the research also successfully classifies these detected human presence into distinct activities in a non-intrusive way. 
Though this work predates both the MAC address randomisation and wide spread use of mobile experienced these days, the explosion of consumer data both publicly available and privately held presents previously unseen opportunity and also limited by the privacy concerns that arise with them.
There is an immense opportunity to collect and standardise a large national level dataset which closely linked to the population distribution and movement in an anonymised way which then can be used to understand the distribution of population and its change.
There is a need to extend such effort  longitudinally which can give us insights in to the change of such phenomenon in time.
This has the potential to enable us to ask broad questions such as,

\begin{itemize}
  \setlength{\itemindent}{2em}
  \itemsep-0.25em
  \item What are the trends in the footfall in UK?
  \item What are the daily rhythm of different cites?
  \item How much a weather event affect economy of a region?
\end{itemize}

Such dataset, in conjunction with other consumer data sources, in addition to augmenting each other to improve their quality, can vastly improve our understanding of the structure and dynamics of population.

%==============================================================================%
\subsection{Device fingerprinting}
%==============================================================================%

\marginnote{\textit{\textbf{Opportunity 2:} Developing models and methods to identify anomalies in the data and underlying events causing them } }

The privacy concerns about the data collected from personal mobile devices has pushed the industry and users to find ways to anonymise the data generate over the last decade.
All the mobility studies recording user trajectories across space and time are rendered infeasible with the cryptographic hashing and randomisation techniques employed by the devices. 
This along with progressive legislation such as General Data Protection Regulation have severely constrained the data available for mobility research.
As we see later, even the estimation of ambient population is limited by these developments.

\citep{vanhoef2016} presents several novel methods of abusing the features of the Wi-Fi standards to track mobile devices even when the MAC addresses were randomised. This research shows the possibility of using the information elements present in the probe requests along with the sequence numbers to fingerprint the mobile device which sent the request with an accuracy of the 50\% within a 20 minute interval with a possibility of improvement with known scrambler 'seeds' - the randomisation factor used by popular commercial devices. Though this sounds promising for short intervals, since this research, manufacturers have stopped including non-mandatory information elements which can affect the accuracy significantly. The research also features two other methods to reverse engineer the original MAC addresses from the randomised ones - first where known hotspots were spoofed to trick the mobile devices in revealing their real addresses and the second where a different protocol requests were used. Both these methods cannot be used extensively since the former is not ethically sound and the latter is not widely used by all mobile devices. 

Since the above study and the following ones were conducted from security perspective - evaluating the robustness of the randomisation/ obfuscation procedure,
they focus on de-anonymising the obfuscated data to recreate the personal information from them while demonstrating vulnerabilities in the standard and associated risks for the users. 
In this context, there is a clear gap for research in to methods to rather carry out  fingerprinting of these devices using patterns in the data to create useful information from them without actually de-anonymise the data.
This can lead to production of data-sets and methodologies which will enable us to,

\begin{itemize}
  \setlength{\itemindent}{2em}
  \itemsep-0.25em
  \item Get accurate estimation of ambient populations.
  \item Understand the movement of the population in space and time.
\end{itemize}

%==============================================================================%
\subsection{Event Detection}
%==============================================================================%

Having granular spatio-temporal data on population at an area level also enables us to look at the activity of people at this scale.
For example, the spike in Wi-Fi activity at a certain area at a certain time can illuminate us with a specific event that is happening in that area.
Thought research have been done on this area using social media data, a longitudinal data-set collected using mobile technology can enable us to formalise the models needed to identify anomalies, quantify the causation of such anomalies to real world event.
\citep{kontokosta2016} discuss the use of Wi-Fi data for a 'real-time' census of the city with a case study of New York City's Lower Manhattan neighborhood.
The research collects around 20 million Wi-Fi data points during 2015 and presents a model to create real-time, on-the-fly population estimates with fine granularity.
The research demonstrates the feasibility of the pursuit along with the potential significance of such localised population estimates for use within the domains of city operations and policy, strategic long-term planning processes, emergency response etc.
There are opportunities to ask questions such as,

\marginnote{\textit{\textbf{Opportunity 3:} Developing models and methods to identify anomalies in the data and underlying events causing them } }

\begin{itemize}
  \setlength{\itemindent}{2em}
  \itemsep-0.25em
  \item How did the tube strike affected London?
  \item What were the hot spots for New years celebration?
  \item What was effect of a road closure in specific part of the city?
\end{itemize}

%==============================================================================%
\subsection{Pedestrian Flow}
%==============================================================================%

Similar to the device fingerprinting, estimating and understanding pedestrian flow in the street network has immense opportunities since the anonymisation of mobile devices has taken off.
Even when the problem of the identifying unique fingerprints of users in the data has not been solved, there is a need to understand the overall performance of the street network in terms of pedestrian flow just from the precise, granular data available, especially when the data source is as unstructured and noisy as the Wi-Fi sensors.

\citep{musa2011} use the Wi-Fi probe requests collected in a 12-hour trial on a busy road to describe a passive tracking system for mobile devices.
The research proposes a trajectory estimation method based on Viterbi's algorithm which estimates the most-likely spatial path taken from the information on when and where they have been detected. 
Although the research extends this trial into a 9-month deployment and demonstrates trajectory estimates with high accuracy, the problem still remains where we need to extract trajectories of users without actually being able to identify them.

This problem can he approached in two ways,

\marginnote{\textit{\textbf{Opportunity 4:} Estimating flow of pedestrians in the street network from Wi-Fi data } }

\begin{enumerate}
  \itemsep-0.25em
  \item Probabilistic approach - Where the relationship between the temporal change in volumes at locations are modelled.
  For e.g. how much and how often the footfall counts at one location mirrors/ follows other location gives us an idea of how many pedestrians move from one location to the other.
  \item Interpolation - Where the relationship between the locations are defined in terms of multiple variables such as how similar they are, how close they are etc.
  These relationships can in turn used to build a graph of locations and use this graph as a source to interpolate other locations.
\end{enumerate}

%==============================================================================%
\subsection{Nature and Change of Places}
%==============================================================================%

Though there are extensive research in using ambient population and people's movement to understand the form and function of the space, the mobile technologies have introduced the opportunity to remove the subjectivity from them.
With access to highly granular and long-term data sets, we can hope to look into the how the places have changed over time and how the external factors such as policy and economy has affected them.
There are opportunities to ask questions such as,

\begin{itemize}
  \itemsep-0.25em
  \item How does UK's exit from EU affect its high streets?
  \item Has a specific area has become more or less vibrant?
\end{itemize}

\marginnote[-3cm]{\textit{\textbf{Opportunity 5:} Using long term data to detect the nature and change of form and function of a place. } }







