\chapter{Visualisations and Applications} \label{chapter:application}

%300 words on how the data can be used.

\section{Footfall Landscape of United Kingdom}
% how the data can be used in understanding the nature of the places they are installed. aggregated to different level.
\subsection{UK footfall index}
% Combining them together to have and index UK. How footfall has gone up or down every week compared to last week or compared to last year.

\subsection{City wise indices}
% The footfall profiles in a day can be used to see how different cities work differently.

\subsection{Location comparisons}
% the locations can vary widely and their profiles can show their nature. compare couple of locations to show difference in character.

\subsection{Longitudinal analysis}
% The profiles can be tracked longitudinally to reveal nature and change of nature over time. Footfall calendar

\section{Events Detection}
% how the data is longitudinal and can be used to detect events from the changes in footfall.

%figure of Cardiff

% Discuss the events

\subsection{football world cup}
% micro site variations could be identified as well.

% match day compared to other days.
% post match celebrations graphic.

\section{Pedestrian Flows}
% Tracks are a problem with this data
% But information can be extracted out of this.
% can use transfer entropy (Roberto)

% Graphic on flow information

% Another approach for interpolation is Geo-propagation
% It can be promising as well.

\section{Functional Hierarchy of Places}

% We can also understand the relationship between the places using this data.
% just using the global MAC address and how much they reoccur between the places we can infer the relationships between them and hence make a judgement of how they are organised.

% Location of 118 sensors in a map.

% network between them along with link widths.

% Hierarchy cluster map.

% Explain how the stuff works or doesn't work.

\section{discussion}

% 500 words on what all the above does and how it can be taken forward.
% emphasize on other work that have been done based off this data

