\chapter{Collecting the Data}

As we saw in the literature search Wi-Fi as a technology is well suited for research into human mobility.
In this chapter we evaluate the technology in detail and devise a methodology to
measure the most possible amount of information from from Wi-Fi.
The data is defined by the IEEE standard.
We start by looking at the standard in detail.
Then we devise simple experiments to collect sample data.
Then we expand the experiments to include locations across London to collect more data.
Finally we give a detailed description of the methodology of the Smart Street Sensor project and the data collected from it.

\section{Wi-Fi as a Source of Data}

Wi-Fi is ubiquitous.
The smart-phone adoption rates are growing.
All smart-phones try to connect to internet.
In addition to mobile networks, Wi-Fi is the second most common way they connect to internet.
Most places provide Wi-Fi as the way to connect to internet.
Unlike mobile networks Wi-Fi is a general purpose service.
There are multiple networks across locations hence the phones are made to be able to move across networks seamlessly.
The mobile phone initiates the contact.
It sends a special signal called - Probe Requests.
This has information about the mobile device.
The router replies with a signal called Probe response.
This forms a digital handshake between these devices.
The devices then carry on with authentication and talking to each other.
After authentication the connection is encrypted and private.
But the probe request process is unencrypted and open.
The probe request is sequential signal which is defined in IEEE standards.
The table shows All possible information that can be included in a probe request.
The figure shows the structure of a probe request.
This is a stream of data broadcast over air from all the phones around a area. 


\section{Initial Experiments}

\section{Pilot Study}

\section{Smart Street Sensor Project}

\section{Uncertainties in Data}

\section{Discussion}
