\chapter{Collecting Wi-Fi Data}

From the literature review in Chapter \ref{chapter:literature}, we observed that  out of all the technologies discussed, Wi-Fi seems to be the most promising one.
We observed the advantages of Wi-Fi based data collection as,

\begin{itemize}
  \item Universality as a standard technology globally,
  \item Independence from other types of data sources or infrastructure,
  \item High level of granularity both spatially and temporally,
  \item Possibility of passive data collection,
  \item Extreme ease of collection in terms of cost and effort and
  \item Scalability to cover study large areas.
\end{itemize}

Though it has its pitfalls in terms of intrusiveness resulting in risk to the privacy of the users, bias and uncertainty, Wi-Fi provides us with a strong base framework for fulfilling the opportunity to design and collect a large, long term and granular dataset which can be used for studying human activity.

In this chapter, we continue our research by looking at Wi-Fi technology closely to understand how it can be used to achieve the aforementioned goal.
We start by looking at the Wi-Fi specification \cite{ieee2016} and focus on the information available within the Wi-Fi probe requests.
We then design and implement a series of data collection exercises which collect probe requests in various location with increasing level of complexity for analysis.
We explore these datasets briefly to understand the usefulness of each set of information present in the probe requests along with the uncertainties in them.
We also introduce `Smart Street Sensor' project - a national scale effort for collecting Wi-Fi data at high streets across United Kingdom.
Finally we summarise the data collection procedure with a detailed look at the each set of uncertainty in these datasets and draw conclusions on further lines of research into alleviating the uncertainty and noise so that they can be used to estimate human activity with confidence.

\section{Wi-Fi as a Source of Data} \label{wifi-as-source-of-data}

Wi-Fi is ubiquitous.
The smart-phone adoption rates are growing.
All smart-phones try to connect to internet.
In addition to mobile networks, Wi-Fi is the second most common way they connect to internet.
Most places provide Wi-Fi as the way to connect to internet.
Unlike mobile networks Wi-Fi is a general purpose service.
There are multiple networks across locations hence the phones are made to be able to move across networks seamlessly.
The mobile phone initiates the contact.
It sends a special signal called - Probe Requests.
This has information about the mobile device.
The router replies with a signal called Probe response.
This forms a digital handshake between these devices.
The devices then carry on with authentication and talking to each other.
After authentication the connection is encrypted and private.
But the probe request process is unencrypted and open.
The probe request is sequential signal which is defined in IEEE standards.
The table shows All possible information that can be included in a probe request.
The figure shows the structure of a probe request.
This is a stream of data broadcast over air from all the phones around a area. 

The probe request frame is the signal sent by a WiFi capable device when it needs to obtain information from another WiFi device.
For example, a smartphone would send a probe request to determine which WiFi access points are within range and suitable for connection.
On receipt of a probe request, an access point sends a probe response frame which contains its capability information, supported data rates, etc.
This ‘request-response’ interaction forms the first step in the connection process between these devices.
The structure of a probe request is shown in Figure 3.3.
We can observe that the request frame has two parts, a MAC header part which identifies the source device and frame body part that contains the information about the source device.
The information that can be included in a probe request shown in Table 3.2.
As mentioned earlier, the SmartStreetSensor collects some of the information present in probe request frame relayed by mobile devices, along with the time interval at which the request was collected and the number of such requests collected during that interval.
The actual information present in the data collected by the SmartStreetSensor is shown in the Table 3.3.

\section{Initial Experiments}

With our theoretical understanding of the Wi-Fi standard and its capabilities, we move on to looking at the Wi-Fi landscape in real-world.
We achieve this by designing small independent experiments where we record the Wi-Fi probe requests within controlled conditions along with the knowledge of the ambient population of the field of measurement. 
We then look at the collected probe requests, examine them in detail to look at their properties, aggregate them to footfall counts and compare them with the real-world counts to get a overall idea of how well they translate into real counts.
The aim of these experiments to know more about the probe requests data and pick out the uncertainties and opportunities present in them.
The objectives here are,

\begin{enumerate}
  \setlength{\itemindent}{2em}
  \itemsep-0.25em
   \item Design a simple method to collect probe requests.
  \item Select locations with different levels of complexity.
  \item Collect real-world data through manual counting.
  \item Analyse the probe requests to extract useful information.
\end{enumerate}

\subsection{Experiment Design}

The first step was to design a simple method to collect Wi-Fi probe requests.
We accomplish by utilising the application - \textit{tshark} \cite{wireshark2} on a regular laptop.
First we put the Wi-Fi module of the laptop in `Monitor mode' where it behaves as a wireless access point.
Then we run tshark to collect the data in CSV format using the following command under a shell.

\begin{minted}{bash}
#! /bin/bash
tshark \
  -I -i en0 \
  -T fields \
  -E separator=, \
  -E quote=d \
    -e frame.time \
    -e frame.len \
    -e wlan_radio.signal_dbm \
    -e wlan_radio.duration \
    -e wlan.sa_resolved \
    -e wlan.seq \
    -e wlan.tag.length \
    -e wlan.ssid \
  type mgt subtype probe-req and broadcast
\end{minted}

The fields marked with \textit{-e} correspond to time stamp when the packet was received, total length of the packet, reported signal strength, duration for which the packet was transmitted, the MAC address of the source device, sequence number of the packet assigned by the source device, length of the tags attached to the packet and the network for which the probe request is being sent for.
The manufacturer information is extracted from the \textit{wlan.sa\_resolved} field into a separate column and the original field is hashed using the SHA256 algorithm implemented in R.
In addition to this, the pedestrians next to the sensor were counted manually by the surveyor.

\subsection{Living room}

\begin{marginfigure}
  \forcerectofloat
  \includegraphics{images/home-total-count.png}
  \caption{Number of probe requests collected every minute on 15 October 2017}
  \label{figure:collection:home:total}
\end{marginfigure}

The first set of experiment was conducted with the laptop in the researcher's living room. 
The room was known to have 2 phones and an Android TV device and the whole house had an occupancy of 3.
Probe requests were collected on 15 Nov 2015 from 19:44 to 21:15 with a gap of 15 minutes in between.
We collected around 3000 probe requests at the rate of 38 requests per minute.
The total number of probe requests collected every minute is shown in Figure \ref{figure:collection:home:total}.

The first thing we looked to convert this to footfall is the mac address along with OUIs.
Initial thoughts on this.
We found that it is not enough.
Strong evidence that mac address randomisation is affecting the numbers.

Other information that can help here - Signal strength, tags, SSID, length, sequence numbers and duration.
Signal strength shows promise!
Length seems useful as it give uniqueness!
Duration and length are directly related and gives no extra information.
SSID and tags have very low or varied data they have no use.
Sequence numbers are interesting and shows most promise.
We need to see the variation and usefulness of length, so need more data.

\subsection{University Hall}
This was conducted to examine the previous results in a larger set of data in a real world setting.
Conducted in one of the main corridors - Southern cloisters of UCL with lot of pedestrian traffic.
There were also seating areas across the corridor where students work with their computers.
The area is also used heavily for lunch for large contingent of visitors.
Collected around 14750 probe requests collected and 652 users were counted walking around the sensors manually.

Major conclusion is that confirmation that Signal strength is still useful.
Length information is not that useful in certain standardised manufacturers.

\subsection{Oxford Street}

Aim is to validate the filtering and clustering methodology against the scale and complexity of data collected in an open public area such as a retail high street.
We also aimed to find the algorithm which was best suited for the classification of signal strengths as 'low' and 'high' in order to filter out the background noise.
The data was collected at Oxford Street, London on 20 December 2017 from 12:30 to 13:00 hrs, Wi-Fi probe requests were collected using the sensor described in Section and pedestrian footfall was manually recorded using the Android app - Clicker bala2018clicker.
Being located at one of the busiest retail locations in the United Kingdom, the Wi-Fi sensor captured approximately 60,000 probe requests during the half hour period; 3,722 people were manually recorded walking on the pavement during that time.
The surveyor positioned himself at the front of a store while carrying the sensor in a backpack and counted people walking by the store on the pavement (3m wide approximately) using a mobile phone.
The sensor was kept as close to the store window as possible, and the manual count was done as a cordon count in front of the store.

The analysis and use of this dataset is 

\subsection{Summary and Discussion}

\section{Pilot Study}

As we see later in Section \ref{section:device-fingerprinting} the efficiency of the methods to clean and aggregate data not only depend on the noise and bias in the data itself but also on external factors such as, the configuration of the sensor in relation to the environment, the day of the week etc.
Thus the dataset captured in our initial experiments, though acts as a good starting point, cannot enable us to generalise our findings to all possible configurations.
This necessitates an even larger data collected over longer durations in challenging situations we usually find in real world conditions.
This was our primary motivation in conducting a pilot study collecting data at 5 locations across London.
The aim was to collect probe requests with information we found relevant in the initial experiments for every location surveyed for at least a full week so that we can understand any artifacts caused by the periodicity of the data.
We also wanted to collect data at all these locations in parallel for at least a week so that they can be compared to one other. 

\subsection{Methodology}

\begin{marginfigure}[2cm]
  \includegraphics{images/pilot-hardware.png}
  \caption{Hardware setup used to collect data in the pilot studies.}
  \label{figure:collection:pilot:hardware}
\end{marginfigure}

The hardware setup for the sensors is illustrated in Figure \ref{figure:collection:pilot:hardware}.
It design of the hardware is not original as it is heavily influenced by the proprietary technology of the data partner for the Smart Street Sensor project albeit a much simpler form. 
The core of the hardware is the general purpose single board computer - Raspberry Pi Model B running Linux Operating system.
Two communication modules - 3G and Wi-Fi were connected to this machine via Universal Serial Bus interface.
3G modem was equipped with a SIM card which it uses to connect to the internet while the Wi-Fi modem is set to 'Monitor' mode.
The board takes power from an outlet and the software is pre installed with the operating system which resides in a Memory card.

The software used for the sensors consists of two parts - sensor software and server software.
The sensor software was written as a mix of Bash script and NodeJS.
Essentially these scripts use wireshark program to capture packets, parse them, anonymises the MAC address fields, adds the location information, encodes them into JavaScript Object Notation format and finally sends it to a server through Web-Socket protocol.
The code used at the sensor side is detailed in Appendix \ref{appendix:sensor:code}.
At the server side we have a similar NodeJS application which listens to the data sent over web sockets, parse them and saves them to a PostgreSQL database.
The server side code is detailed in Appendix \ref{} and schematic diagram for the whole process is shown in Figure \ref{figure:collection:pilot:schema}.
The information collected from each probe request at these locations are,

\begin{enumerate}[leftmargin=4em, rightmargin=2em]
  \itemsep-0.25em
  \item Time stamp at which it was received
  \item MAC address of the source device.
  \item Signal Strength of the packet.
  \item Total length of the packet.
  \item Sequence number of the packet.
  \item OUI part of MAC address.
  \item Location at which it is collected.
\end{enumerate}

\begin{figure*}
  \includegraphics{images/pilot-study-system.jpeg}
  \caption{System diagram showing the data collection process in the pilot study.}
  \label{figure:collection:pilot:schema}
\end{figure*}

\subsection{Locations}
Locations where sensors were installed, volume and speed of probe requests collected by the sensor and total pedestrians manually counted.
The data occupies around 1.8 GB on disk when encoded in text format.
The locations at which the data were collected are shown in Table .
The locations were chosen for their diverse site conditions and unique sources of noise around the potential location of the sensors.
The position of the sensor at these locations with respect to the context is shown the Figure 
We can see that Location 5 is the `cleanest' with one clear stationary source of noise (phone shop) while location 2 is the most complex due to the proximity of seating areas to the sensor.
The sensors were operational through out February and March, while manual counts were conducted in these locations in half hour sessions on at least two different days.
For the purposes of comparing with ground truth, we considered the data from sensors which correspond to the 12 sets of available manual counts.
The schedule of data collection is shown in Figure .

\begin{figure}
  \centering
  \includegraphics[trim={20 20 20 20},clip, width=\textwidth]{images/pilot-study-locations.png}
  \caption{Pilot study locations in London along with their corresponding sensor installation configurations.}
  \label{figure:literature:tech:timeline}
\end{figure}

\subsection{Data Collection}

\begin{figure*}
  \includegraphics{images/pilot-study-schedule.png}
  \caption{Outline of the `Medium data toolkit' devised to collect, process, visualise and manage the Wi-Fi probe requests data}
  \label{figure:literature:tech:timeline}
\end{figure*}

\lipsum[1]

\begin{table}
  \footnotesize
  \begin{center}
    \begin{tabular}{clllrr}
      \toprule
        Id & Location & Type & Installation notes & Probes\textsuperscript{*} & Footfall\textsuperscript{**}\\
      \midrule
        \addlinespace[0.2cm]
        1 & Camden St. & Phone Shop & Bus stop in front & 9.9 (297) & 3683 (33)\\
        \addlinespace[0.1cm]
        2 & St.Giles & Restaurant & Seating on both sides & 3.9 (169) & 0346 (05)\\
        \addlinespace[0.1cm]
        3 & Holborn Stn. & Info. Kiosk & Front of station entrance & 4.3 (303) & 2956 (46)\\
        \addlinespace[0.1cm]
        4 & Brunswick & Fast Food & Seating  on one side & 3.4 (210) & 0960 (12)\\
        \addlinespace[0.1cm]
        5 & The Strand & Tea Shop & Phone shop next door & 8.4 (382) & 1969 (21)\\
        \addlinespace[0.05cm]
      \bottomrule
    \end{tabular}
  \end{center}
  \caption{Locations of data collection in the pilot study and the amount of data collected at each locations}
  \label{table:collection:pilot:location}
\end{table}
\marginnote[-1.5cm]{\textit{*something  **somethingelse}}

\section{Smart Street Sensor Project}

The Smart Street Sensor project is one of the most comprehensive study carried out on consumer volume and characteristics in retail areas across UK.
The project has been organised as a collaboration between Local Data Company (LDC) and Consumer Data Research Centre,  University College London (CDRC, UCL).
The data for the study is generated independently within the project through sensors installed at around 1000 locations across UK.
When completed, the project will serve as the first and unique comprehensive research into the patterns of retail activity in UK high streets.

The primary aim of the project is to improve our understanding of the dynamics of the high street retail in UK.
As we saw in our literature search, unlike online retail, this involves quantification and measurement of human activity at small scales, such as high streets which already the subject of active research.
The key challenge in this area is the collection of data at smallest scales possible with minimal resources while not infringing on people’s privacy.
This challenge when solved can provide immense value to occupiers, landlords, local authorities, investors and consumers within the retail industry.
The project aims to facilitate decision making by stakeholders in addition to the tremendous opportunities for academic research.

\subsection{Methodology}

As a first step, various locations for the study were identified by CDRC to include a wide geographical spread, different demographic characteristics and range of retail centre profiles.
A custom footfall counting technology using WiFi based sensors was also designed, developed by LDC and the sensors were installed the identified locations.
The sensor monitors and records signals sent by WiFi enabled mobile devices present in its range.
In addition, the number of people walking by the sensor was counted manually for short time periods during the installation.
The project aims to combine these two sets of data to use as a proxy for estimating footfall at these locations.
The potentially identifiable information collected on the mobile devices is hashed at sensor level and the data is sent to central server via encrypted channel for storage.
This data is then retrieved securely for the preparation of the commercial dashboards by LDC and for research purposes by CDRC users.
The project began on July 2015 with the first sensor installation and has grown to an average of 450 daily active sensors as of January 2017.

\begin{figure*}
  \includegraphics{images/sss.png}
  \caption{Outline of the `Medium data toolkit' devised to collect, process, visualise and manage the Wi-Fi probe requests data}
  \label{figure:literature:tech:timeline}
\end{figure*}


The data is collected through set of SmartStreetSensors (shown in Figure 3.1), a WiFi based sensor which when installed acts as a WiFi access point and collects specific type of packets (probe requests) relayed by mobile devices which are which are within the device’s signal range and are searching for available access points.
The sensor is usually installed on partnering retailer's shop windows so that its range covers the pavement in front of the shops.
The installation and calibration of device with respect to the shop window and the pavement is illustrated in Figure 3.2.
There is also a small percentage (3\%) of the devices which are installed within large shops to monitor internal footfall.
Each device collects data independently and uploads the collected data to a central container at regular interval 5 minutes through a dedicated 3G mobile data connection.
The sensor hardware has been improved over the course of the project and currently has built in failure prevention mechanisms such as, backup battery for power failures, automatic reboot capabilities and in-device memory for holding data when internet is not available.
The hardware versions and the corresponding features are detailed in Table 3.1.
%+200 words
smartstreetsensor - figure.
system architecture - figure.

\subsection{Locations}
%(300 words)
locations - map
locations - table

\subsection{Data Description}
%(300 words)
data description - table.
sample probe request - code.
installation of sensors - figure.
descriptive statistics - table.

\input{chapters/34__discussion.tex}
