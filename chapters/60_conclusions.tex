%==============================================================================%
\chapter{Discussion and Conclusions}
%==============================================================================%
In the past 30 years there have been an immense change in the way data concerting distribution and dynamics of human population are generated, collected and consumed.
Rather than being a top-down, structured designed endeavour, data generation and collection has become a bottom-up procedure where data were created as a result of day to day activities of people and are collected, cleaned, and  aggregated into information.
There have been a significant volume of research on identifying such data sources and use them for various purposes in both academia and industry.
As these data sources got more distributed and comprehensive, the concern to protect user privacy increased as well.
This thesis aims to work in between these two the areas of research and the corresponding opportunities present in them.
This thesis identifies Wi-Fi probe requests as a source of data from which information on ambient population and behaviour - especially footfall - could be extracted, and solves the problem of inferring accurate footfall information without using personally identifiable information of the users.

%------------------------------------------------------------------------------%
\section{Summary of Findings}
%------------------------------------------------------------------------------%

From the systematic literature review of around 350 academic publications, it was concluded that Wi-Fi is the most suitable candidate for the technology that can be used to collect data on human presence and movement at a national level.
It was found to be the scalable, cheap, universal, and easy way to collect large amounts of granular such data without depending on any other infrastructure.
The only shortcomings of projects using Wi-Fi technology are their inherent uncertainties and the leakage of MAC addresses - a globally unique, personally identifiable information which could be related back to the users relatively easily.
Finally, two potential areas of research with opportunities for further study were identified:  Creating a standardised, cleaned, accurate and reliable footfall or ambient population from the Wi-Fi probe requests, and solving the specific issue of MAC address randomisation while cleaning and filtering Wi-Fi probe requests. 
After studying the Wi-Fi specification to get a overall outline of the structure of Wi-Fi probe request, a set of small initial experiments were designed and executed to know more about them.
It was found that the number of probe requests and the unique MAC addresses collected by a Wi-Fi sensor is far greater than the number of mobile devices in the immediate vicinity.
The signal strength and sequence numbers are some of the important information that are present in the probe requests.
Alongside these experiments a longer pilot study was also conducted to result in three distinct datasets.
The small experiments collected in-depth data on probe requests at small areas for short intervals.
The pilot study covered 5 locations across central London collected data for over two and a half months.
Finally the Smart Street Sensor project which collected small set of data from probe requests at 1000 locations continuously for over 3.5 years.

Before moving to cleaning and processing the data, this research undertook a comprehensive look at the nature of these datasets within the context of 'Big data and Big data tools' so that a framework for evaluating the 'bigness' of the datasets can be devised.
With such a framework the Wi-Fi datasets were examined in all of their dimensions and found to be 'Medium data' at best.
A review of big data tools was also carried out and the tools suitable for the Wi-Fi dataset were picked out and combined together to form a bespoke 'Medium data toolkit' for processing the Wi-Fi data as efficiently as possible.

From the initial exploration, the major uncertainties identified in the data which needs to be solved by the cleaning and processing procedures are, range of the Wi-Fi sensor, differing frequency at which mobile devices emit probe requests, MAC address randomisation which masks the devices unique identification, increasing mobile devices ownership in the population over long-term, missing data from the failures in the sensors and collisions of MAC addresses when they are anonymised using cryptographic hashing.
The collisions in hashed MAC addresses were found to be rare and inconsequential.The uncertainty regarding the range of the Wi-Fi sensor was found to cause noise in the data from outside the field of measurement and was solved by filtering out probe requests with low signal strength.
This definition of 'Low' signal strength could be deduced dynamically for each location at each time interval using one dimensional clustering algorithms.
The 'k-means' algorithm was found to be best suited for this purpose.
The randomisation of MAC addresses lead to over-estimation of number of devices from set of probe requests while the uneven frequency of probe requests emitted by the mobile devices prohibit a simple universal factor for converting number of probe requests to number of devices.
It was found that this uncertainty can be solved using a novel graph based methodology which uses the sequence numbers in the packets rather than the MAC address to uniquely identify the devices.
When the sequence number is not available the uncertainty can be reduced for an interval by looking at the ratio of number of probe requests to the number of mobile devices in the probe requests without randomised addresses in that interval.
These methods along with manual calibration were found to reduce the error in the estimation of footfall from Wi-Fi probe requests to almost 10\% at locations with ideal conditions.

In addition to the above data cleaning techniques other processing were done to the probe requests dataset to remove further uncertainties.
The missing data in the dataset could be interpolated using a Kalman smoothing based method for short term or a seasonally decomposed method for long term.
Finally, the increasing mobile ownership was adjusted using manual counts for short term intervals and using a weekly adjustment factor 0.2\% for long term.
Using all these methods for filtering, cleaning and adjusting Wi-Fi data, this thesis finalises a overall data processing pipeline for producing a clean, precise, accurate and continuous data on footfall across retail locations across the UK.
Finally this research also provides a gallery of examples showing the possible use of such granular and continuous data on footfall on a national level.

%------------------------------------------------------------------------------%
\section{Research Question}
%------------------------------------------------------------------------------%
Looking back at our research question - "Can dynamics of footfall inferred from passively collected big dataset without putting the privacy of users at risk?",
we have demonstrated that the task is indeed feasible, using Wi-Fi probe requests. Even when the identity of the devices were masked using randomisation techniques we have demonstrated that aggregation and estimation could be done without compromising the privacy of the users.
In addition to this, we have also demonstrated the usefulness and application of such footfall estimate with various examples.
The footfall estimates derived from the method were used to devise a 'footfall index' at various levels - national, city, area and micro site locations showing how the retail related footfall have been distributed in the UK and how this distribution has been changing over time in high granularity both spatially and temporally.
It has been demonstrated that this information on footfall can be used as a clue for knowing the form and function of a place and trace the changes it has undergone over time as well.
It was also demonstrated two sets of examples, that real-world events could be detected from looking at the anomalies in the footfall volumes at locations.
Finally, it was also demonstrated that such detailed and continuous footfall volume information at locations could be used to predict or estimate flow of pedestrians between them by just looking at the changes in these volumes thus providing a way to understand the pedestrian flow in cities without actually tracking individuals.

%------------------------------------------------------------------------------%
\section{Further Work}
%------------------------------------------------------------------------------%

As we discussed in the literature review, the research on collecting and using data on population distribution and dynamics have closely followed the advances and changes in the consumer technology. 
Every new technology adopted for mainstream use spurred new wave of research in using those technology.
It is also noted that every new technology not only brought many advantages over the previous ones but also introduced unique challenges.
In this context, the larges opportunity in furthering the research exists in identifying, evaluating and adopting new technologies.
There is a significant opportunities in applying these new technologies for old challenges and device methods to make them suitable to answer the questions raised by research.
Few such technologies are detailed below,

\begin{itemize}
  \item \textbf{5G} is the new generation of technology which aims to bring even higher speeds of data transfer to mobile devices through cellular networks. This may lead to the gradual decline and phasing out of Wi-Fi technology. Though this cellular based technology doesn't provide the similar detail and flexibility offered by Wi-Fi it has the potential to offer much more comprehensive picture of the world if it gets  widely adopted.
  \item \textbf{Bluetooth Low energy (BLE)} is the upcoming short-range, wireless personal area network technology. With emphasis on being the technology used by the Internet of Things (IOT) devices, this technology has the potential to displace Wi-Fi as the choice of short-range communications. The explosion of wearables and smart devices at home, the amount of data that could be available form this technology could be staggering in the next decade.
  \item \textbf{Ultra wide band radar} is another short-range technology which has been developed for motion and object detection. Being primarily used to design sensors for proximity and motion detection, this has the potential to become a standard for vehicles. Moreover, with the recent uptick in self driving car research and development, the cost of these devices has gone significantly down thus providing amazing opportunities in creating comprehensive sensor networks similar to Smart Street Sensor project.
\end{itemize}

In spite of being developed since 1980s, machine learning techniques have received extraordinary interest in the last decade.
This interest, along with advancements in the Big data tools and technologies has set up the stage for research by applying supervised and unsupervised machine learning techniques on large scale datasets collected through the above mentioned technologies.
There is a significant opportunity for applying unsupervised learning techniques such as anomaly detection and neural networks in passively collected digital data to improve data cleaning, interpolation, population estimations and time series based predictions etc.

Research ethics, safety and privacy are going to be the next big areas of concern for advanced machine learning based techniques and big data analysis in the next decade.
The era of uninhibited large scale production, collection and consumption of personal data through connected devices over internet without oversight is almost over.
People are increasingly concerned with protecting their privacy and are opposed to the exploitation of their personal data.
This concern has been addressed by legislation such as GDPR and technologies such as cryptography and randomisation.
All these developments provide us with various opportunities in further research.

Firstly there is opportunity study the above mentioned technologies form a privacy point of view to evaluate the advantages and risks presented by them and advance the research in terms of both mitigation the risks while maintaining some kind of usefulness.
These inquiries can not only be done in terms of techniques but also on the lines of legal compliance of such techniques. 
There is also opportunity for researching on the uncertainties and limits of datasets when subject to robust privacy control methods.
Secondly the immense research, innovation and advancements made in peer to peer technologies in solving the various trust problems could be applied in the field of sensor based population estimation or pedestrian flow detection.
There is an opportunity for research into building a peer to peer network of sensors where the data collected by the sensors never leave the device themselves but the analyses are taken to the source of data. 
This act of "moving the analysis to data" can solve numerous problems of safety of the personal data since there is not central point of failure and it can also scale up indefinitely without overwhelming a central repository of data.
Through these further research, we could take the field forward by not only following the improvements in the technology of data collection but also push the envelope in terms of developing more ethical and sage research environment while handling large amounts of data.
